\documentclass[10pt, a4paper]{extarticle}

%% Язык
\usepackage{cmap} % Поиск в PDF
\usepackage{mathtext} % Кириллица в формулах
\usepackage[T2A]{fontenc} % Кодировка
\usepackage[utf8]{inputenc} % Кодировка
\usepackage[english,russian]{babel} % Локализация, переносы

%% Шрифты

% Serif
%\usepackage{euscript} % Шрифт Евклид
%\usepackage{mathrsfs} % Шрифт для математики
\usepackage{libertinus}

% Sans-serif
%\renewcommand{\rmdefault}{cmss}
\renewcommand{\ttdefault}{cmss}
%\usepackage{sfmath}

% Настройки для xelatex
%\usepackage{polyglossia} % Для выбора языка в xelatex
%\setmainlanguage{russian}
%\setotherlanguages{english}
% Ligatures=TeX is on by default
% https://tex.stackexchange.com/questions/323542/
%\setmainfont[Ligatures=TeX]{Cantarell}
%\newfontfamily{\cyrillicfonttt}{Times New Roman}
%\newfontfamily\cyrillicfont{Cantarell}[Script=Cyrillic]
%\setsansfont[Ligatures=TeX]{Cantarell}
%\newfontfamily\cyrillicfontsf{Cantarell}[Script=Cyrillic]
%\setmonofont{Courier New}
%\newfontfamily\cyrillicfonttt{Courier New}[Script=Cyrillic]

%% Математика
\usepackage{amsmath, amsfonts, amssymb, amsthm, mathtools}
\usepackage{icomma}

% Операторы
\DeclareMathOperator*\plim{plim}
\DeclareMathOperator{\sgn}{sign}
\DeclareMathOperator{\sign}{sign}
\DeclareMathOperator*{\argmin}{arg\,min}
\DeclareMathOperator*{\argmax}{arg\,max}
\DeclareMathOperator*{\amn}{arg\,min}
\DeclareMathOperator*{\amx}{arg\,max}
\DeclareMathOperator{\cov}{Cov}
\DeclareMathOperator{\Var}{Var}
\DeclareMathOperator{\Cov}{Cov}
\DeclareMathOperator{\Corr}{Corr}
\DeclareMathOperator{\pCorr}{pCorr}
\DeclareMathOperator{\E}{\mathbb{E}}
\let\P\relax
\DeclareMathOperator{\P}{\mathbb{P}}
\renewcommand{\le}{\leqslant}
\renewcommand{\ge}{\geqslant}
\renewcommand{\leq}{\leqslant}
\renewcommand{\geq}{\geqslant}

% Распределения
\newcommand{\cN}{\mathcal{N}}
\newcommand{\cU}{\mathcal{U}}
\newcommand{\cBinom}{\mathcal{Binom}}
\newcommand{\cPois}{\mathcal{Pois}}
\newcommand{\cBeta}{\mathcal{Beta}}
\newcommand{\cGamma}{\mathcal{Gamma}}

% Множества
\def \R{\mathbb{R}}
\def \N{\mathbb{N}}
\def \Z{\mathbb{Z}}

% Другое
\newcommand{\dx}[1]{\,\mathrm{d}#1} % Для интеграла: маленький отступ и прямая d
\newcommand{\ind}[1]{\mathbbm{1}_{\{#1\}}} % Индикатор события
\newcommand{\iid}{\mathrel{\stackrel{\rm i.\,i.\,d.}\sim}}
\newcommand{\const}{\mathrm{const}}

%% Изображения
\usepackage{graphicx}
\usepackage{caption}
\usepackage{subcaption}
\usepackage{physics}
\usepackage{wrapfig} % Обтекание рисунков и таблиц текстом
\usepackage{tikz}

%% Таблицы
\usepackage{array, tabularx, tabulary, booktabs}
\usepackage{longtable}  % Длинные таблицы
\usepackage{multirow} % Слияние строк в таблице

%% Cписки
\usepackage{multicol}
\usepackage{enumitem}

%% Гиперссылки
\usepackage{xcolor}
\usepackage{hyperref}
\definecolor{linkcolor}{HTML}{8b00ff}
\hypersetup{colorlinks = true,
			linkcolor = linkcolor,
			urlcolor = linkcolor,
			citecolor = linkcolor}

%% Выравнивание
\setlength{\parskip}{0.5em} % Расстояние между абзацами
\usepackage{geometry} % Поля
\geometry{
	a4paper,
	left=20mm,
	top=20mm,
	right=20mm}
\setlength{\parindent}{0cm} % Отступ (красная строка)
\linespread{1.0} % Интерлиньяж
\usepackage[many]{tcolorbox}  

%% Оформление

\newtcolorbox{rulesbox}[1]{%
	tikznode boxed title,
	enhanced,
	arc=0mm,
	interior style={white},
	attach boxed title to top center= {yshift=-\tcboxedtitleheight/2},
	fonttitle=\bfseries,
	colbacktitle=white,coltitle=black,
	boxed title style={size=normal,colframe=white,boxrule=0pt},
	title={#1}}

% Красивый серый фон
\usepackage{framed} 
\definecolor{shadecolor}{gray}{0.9}

% Код
\newcommand{\code}[1]{\colorbox{lightgray}{{\tt #1}}}

% Колонтитулы
\usepackage{fancyhdr}
\pagestyle{fancy}
\fancyhf{}
\fancyhead[L]{}
\fancyhead[R]{\thepage}

% Разделы и подразделы
\usepackage[sf, sl, outermarks]{titlesec}
\titleformat{\section}{\Large\bfseries\sffamily}{\thesection}{0.5em}{}
\titleformat{\subsection}{\large\sffamily}{\thesubsection}{0.5em}{}

% Содержание
%\usepackage{tocloft}
%\renewcommand{\cftsecfont}{\hspace{4.5em}\normalfont}
%\renewcommand{\cftsubsecfont}{\hspace{5em}\normalfont}
%\renewcommand{\cftsecpagefont}{\normalfont\hfill}
%\renewcommand{\cfttoctitlefont}{\large\normalfont\hfill}
%\renewcommand{\cftaftertoctitle}{\hfill}
%\renewcommand{\cftsecleader}{\cftdotfill{\cftdotsep}}
%\renewcommand{\cftsecafterpnum}{\hspace*{5.5em}\hfill}
%\renewcommand{\cftsubsecafterpnum}{\hspace*{5.5em}\hfill}
%\renewcommand{\cftsecaftersnum}{.}
%\renewcommand{\cftsubsecaftersnum}{.}

%% Комментарии
\usepackage{comment}

%% To-do
\usepackage{todonotes}

%% Литература
\usepackage[backend = biber,
			bibencoding = utf8, 
			sorting = nty, 
			maxcitenames = 4,
			style = numeric-verb]{biblatex}
\addbibresource{lit.bib}
\usepackage{csquotes}

%% Заголовок
\title{{\normalsize Интеллектуальный анализ данных 2021 – Группа ИАД-5} \\ \vspace{0.5em} Консультация по Git и GitHub}
\author{\rule{15cm}{0.4pt}}




\begin{document}
	
	\maketitle

	{\Large \textbf{Перед консультацией.}}
	
	\begin{enumerate}
		\item Установите Git на ваш компьютер.
		\begin{description}
			
			\item[Win] установите Git for Windows Setup \href{http://git-scm.com/download/win}{с официального сайта}. В окне «Adjusting the name of the initial branch in the new repositories» выберите «Let Git decide». В окне «Choose the default behavior of git pull» выберите «Default». В окне «Choose a credential helper» выберите «Git Credential Manager Core». В окне «Configuring experimental options» не выбирайте ничего. Проверьте, что при запуске программы Git Bash открывается тёмное окно с разноцветными надписями. 
			
			\item[Mac] откройте приложение «Терминал». Введите любую команду, связанную с Git, например, \code{git status}. Откроется системное окно, в котором будет предложено установить дополнительные инструменты разработчика XCode. Согласитесь и установите их. 
			
			\textbf{NB!}: сам XCode устанавливать не требуется, достаточно установить только дополнительные инструменты разработчика, как описано выше. 
		\end{description}
	
		\item Зарегистрируйтесь на \href{https://github.com}{GitHub}.
	\end{enumerate}
	
	\vspace{1em}
	
	{\Large \textbf{Создание и управление репозиторием.}}
	
	\begin{enumerate}
		\item Команда \code{git init} позволяет создавать новые репозитории. Создайте папку {\tt git\_cons} в домашней папке и локальный репозиторий в ней.
		
		\textbf{NB!}: для перехода между папками в терминале можно использовать команду \code{cd}.
		
		\textbf{NB!}: кнопка {\tt TAB} позволяет использовать автодополнение.
		
		\item Создайте в репозитории текстовый файл {\tt test.txt} с двумя (любыми) строками.
		\item Команды:
			\begin{itemize}
				\item \code{git add <filename>} добавляет изменения в staging area.
				\item \code{git commit -m '<message>'} создаёт новый commit в репозитории. 
			\end{itemize}
		Сделайте commit в репозиторий. Введите какое-нибудь осознанное сообщение.
		
		\textbf{NB!}: {\tt .} означает «все файлы».
		
		\item Команда \code{git status} позволяет узнать, что происходит в репозитории. Используйте эту команду и прокомментируйте выдачу. Внесите какие-нибудь изменения в текстовый файл. Что происходит в репозитории теперь? Сделайте commit с вашими изменениями.
		\item Команда \code{git help <command>} позволяет получить справку о команде. С помощью неё выясните, как создать новую ветку командой \code{git checkout}.
		\item Выясните, в какой ветке вы находитесь при помощи команды \code{git branch}. Создайте новую ветку {\tt git\_test} при помощи команды из предыдущего пункта.
		\item Выясните, в какой ветке вы находитесь. Внесите в текстовый файл какие-нибудь изменения и сделайте commit в ветке {\tt git\_test}. Убедитесь, что файл в основной ветке не был изменён. 
	\end{enumerate}
	
	\vspace{1em}
	
	{\Large \textbf{Взаимодействие с GitHub.}}
	
	\begin{enumerate}
		\item Создайте новый удалённый публичный репозиторий на GitHub и назовите его {\tt git\_cons}. При помощи команды:
		
		\code{git remote add origin <path>},
	
		добавьте удалённый репозиторий для вашего локального репозитория.
		
		\textbf{NB!}: название удалённого репозитория можно посмотреть при помощи команды \code{git remote}. В данном случае мы назвали его {\tt origin}.
		
		\item Перейдите на основную ветку (далее будем работать только с ней). Отправьте (push) локальные изменения на удалённый репозиторий при помощи команды \code{git push origin <main branch>}. Проверьте, что изменения появились на GitHub. Отправьте локальные изменения ветки {\tt git\_test}.
		\item Измените текстовый файл вручную на GitHub. Синхронизируйте изменения на локальный компьютер при помощи команды \code{git pull origin <branch>}. Проверьте, что локальный файл изменился.
		\item Теперь измените файл и на компьютере, и на GitHub так, чтобы изменения были противоречивыми. Сделайте commit в обоих местах. Используйте команду \code{git pull origin <branch>}. Устраните возникший конфликт. 
		\item Клонируйте \href{https://github.com/V-Marco/hse_iad5_2021}{репозиторий нашей группы} на локальный компьютер при помощи команды \code{git clone <remote>}.
		\item Сделайте fork репозитория нашей группы и клонируйте его на компьютер. Локально добавьте в него какой-нибудь файл и сделайте commit и push. 
		\item Создайте pull request для внесения ваших изменений в репозиторий нашей группы. 
	\end{enumerate}

	\vspace{1em}
	
	{\Large \textbf{Полезные ссылки.}}
	
	\begin{enumerate}
		\item \href{https://git-scm.com/docs}{Документация Git.}
		\item \href{https://dzone.com/articles/top-20-git-commands-with-examples}{Команды с описанием.}
		\item \href{https://desktop.github.com}{GitHub Desktop.}
	\end{enumerate}
	
\end{document}